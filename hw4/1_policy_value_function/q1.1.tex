
\paragraph{}
In this context:

\begin{align*}
    r_t(S_1, \text{"stay"}) &= -1\\
    r_t(S_2, \text{"stay"}) &= -1
\end{align*}

In other word, no matter the first state, we can simplify the equation in the following way:

\begin{align*}
    \sum^\infty_{t=0} \gamma^t r_t(s_t, a_t) &= \sum^\infty_{t=0} - \gamma^t\\
    &= - lim_{t \rightarrow \infty} \left( \frac{\gamma^t - 1}{\gamma - 1} \right) \text{ if } \gamma \neq 1
\end{align*}

In this question $\gamma$ is described as a discount, which is usually so $\gamma \in [0, 1[$.
If it's the case you can directly look in Section \ref{q1.1res} for the result using $\gamma$ in the classical range.
But this range is not precised on the question we will so explore all the different cases.

\subsubsection{$\gamma \in ]-1, 1[$} \label{q1.1res}

\begin{align*}
    \sum^\infty_{t=0} \gamma^t r_t(s_t, a_t) &= \frac{1}{\gamma - 1}
\end{align*}

\subsubsection{$\gamma = 1$}

\begin{align*}
    \sum^\infty_{t=0} \gamma^t r_t(s_t, a_t) &= \sum^\infty_{t=0} - 1^t\\
    &= - \infty
\end{align*}

\subsubsection{$\gamma \in ]1, +\infty[ $}

\begin{align*}
    \sum^\infty_{t=0} \gamma^t r_t(s_t, a_t)
    &= -lim_{t \rightarrow \infty} \left( \frac{\gamma^t - 1}{\gamma - 1} \right)\\
    &= - \infty
\end{align*}

\subsubsection{$\gamma = -1$}

In this case $\sum^\infty_{t=0} \gamma^t r_t(s_t, a_t)$ does not converge and just alternate between $0$ and $1$.

\subsubsection{$\gamma \in ]-\infty, -1[ $}

\begin{align*}
    \sum^\infty_{t=0} - \gamma^t &= - \sum^\infty_{t=0} \gamma^{2t} + \gamma^{2t+1} \\
    &= -lim_{t \rightarrow \infty} \left( \frac{{\gamma^2}^t - 1}{\gamma^2 - 1} + \gamma \frac{{\gamma^2}^t - 1}{\gamma^2 - 1} \right) \\
    &= -lim_{t \rightarrow \infty} \left( (\gamma + 1) \frac{{\gamma^2}^t - 1}{\gamma^2 - 1} \right) \\
    &= + \infty
\end{align*}


