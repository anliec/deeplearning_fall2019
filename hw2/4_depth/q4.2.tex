\paragraph{}
If we only consider the path that seams to be wanted by the question, $x$ in $]0,1[^d$, $g(x)$ in $]0,1[^d$ and
$f \circ g (x)$ in $]0,1[^d$.
Then it easily appear that each identified regions by $g$ would be split into $n_f$ regions.
Leading to the fact that $f \circ g (x)$ identify onto $]0,1[^d$ $n_g \cdot n_f$ regions.
Which look like the expected answer.

It's important to note that this implication is only guarantee if each of the $n_g$ regions of $g$ identify to
$]0,1[^d$ and no one of its subset like $]0,0.1[^d$.
Such behaviour would lead to some input regions of $f$ not being hit, and so, some output regions may be missed, leading
us to the fact that under this current set of constraint on $f$ and $g$ and the current definition of identification,
the $n_g \cdot n_f$ number of regions is an upper bound of the final results.

\paragraph{}
In addition, counting the number of regions on which $f \circ g (\cdot)$ identify onto $]0,1[^d$, doesn't imply any
assumptions on the intermediate results of function $g$.
For example function $g$ may identify $n_{g2}$ regions of $]0,1[^d$ onto $]1,2[^d$ and $f$ $n_{f2}$ regions of $]1,2[^d$
onto $]0,1[^d$.
This would increase the final results by $n_{g2} \cdot n_{f2}$.

We could apply such a reasoning on every none overlapping intervals where $f$ inputs and $g$ outputs are defined.
However such definition of $f$ and $g$ are not given here, so we can't assume that such path doesn't exists.
This show that the final result may be higher than the previously given $n_g \cdot n_f$.

\paragraph{}
In conclusion as stated previously, the answer to this question is the product of the regions of each sub function,
in this case $n_g \cdot n_f$.
But this imply a set large number of assumption that are not defined, and so may not hold true.
Depending on the assumptions you take this number can be a lower bound or a upper bound, showing that we can't get a
precise number for the general case.


