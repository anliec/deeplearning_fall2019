
\paragraph{}
Following the homework instructions, we are looking $W$ and $b$ such as:

\[
    [y_t] = act\left( W \cdot
    \begin{bmatrix}
        y_{t-1} \\
        x_t
    \end{bmatrix}
    + b
    \right)
\]

\paragraph{}
Following the instruction in the instructions, we can build the following truth table:

\begin{table}[h!]
    \begin{center}
        \begin{tabular}{ll|ll}
            & \multicolumn{1}{c}{}  & \multicolumn{2}{c}{$y_{t-1}$} \\
            &   & 0 & 1 \\ \cline{2-4}
            \multirow{2}{*}{$x_t$}
            & 0 & 0 & 1 \\
            & 1 & 1 & 0 \\
        \end{tabular}
        \caption{Truth table of $y_t$}
    \end{center}
\end{table}

\paragraph{}
This table is basically the truth table of an XOR function.
We saw during the last homework that such a function can't be represented by a single neurone and a linear activation.

\paragraph{}
Hopefully we are here free to use the activation function we want.
If a linear activation is unable to represent XOR, an absolute activation function is a good fit.

Using the information from above we can quickly our neurone as:

\[
    [y_t] = abs\left(
    \begin{bmatrix}
        1 & -1 \\
    \end{bmatrix}
    \cdot
    \begin{bmatrix}
        y_{t-1} \\
        x_t
    \end{bmatrix}
    + 0
    \right)
\]
\paragraph{}
We can now apply this set of weight to the four possible incomes, as displayed on Table \ref{neuroneValue}

\begin{table}
    \begin{center}
        \begin{tabular}{ll|ll}
            & \multicolumn{1}{c}{}  & \multicolumn{2}{c}{$y_{t-1}$} \\
            &   & 0 & 1 \\ \cline{2-4}
            \multirow{2}{*}{$x_t$}
            & 0 & 0  & 1 \\
            & 1 & -1 & 0 \\
        \end{tabular}
        \caption{Value table of our neurone for all possible inputs}
        \label{neuroneValue}
    \end{center}
\end{table}
\paragraph{}
If we then apply the activation function we chose to the value of Table \ref{neuroneValue}, we easily fall back on the truth table of XOR, as showed on Table \ref{neuroneValueFinal}.

\begin{table}
    \begin{center}
        \begin{tabular}{ll|ll}
            & \multicolumn{1}{c}{}  & \multicolumn{2}{c}{$y_{t-1}$} \\
            &   & 0 & 1 \\ \cline{2-4}
            \multirow{2}{*}{$x_t$}
            & 0 & 0 & 1 \\
            & 1 & 1 & 0 \\
        \end{tabular}
        \caption{Value table output of our neurone for all possible inputs with the preset weights and bias as well as the selected activation function}
        \label{neuroneValueFinal}
    \end{center}
\end{table}

The above results assume that $h$ is initialized with a value of zero.
